% figures/architecture.tex
% SOKRATES architecture diagram using TikZ

\begin{figure*}[t]
\centering
\resizebox{\textwidth}{!}{%
\begin{tikzpicture}[
    % Node styles
    data/.style={rectangle, draw=black!70, fill=blue!8, minimum width=2cm, minimum height=0.8cm, align=center, font=\small},
    process/.style={rectangle, draw=black!70, fill=orange!12, minimum width=2.2cm, minimum height=0.8cm, align=center, font=\small, rounded corners=2pt},
    model/.style={rectangle, draw=black!70, fill=green!10, minimum width=2cm, minimum height=0.8cm, align=center, font=\small, rounded corners=2pt},
    output/.style={rectangle, draw=black!70, fill=purple!10, minimum width=2.2cm, minimum height=0.8cm, align=center, font=\small, rounded corners=2pt},
    loopbox/.style={rectangle, draw=black!60, fill=yellow!5, rounded corners=5pt, inner sep=8pt},
    arrow/.style={->, >=stealth, thick, black!70},
    looparrow/.style={->, >=stealth, thick, black!50, dashed},
    label/.style={font=\footnotesize\itshape, text=black!60},
]

% === LEFT: Data Sources ===
\node[data] (prontoqa) at (-0.4, 1.0) {PrOntoQA};
\node[data] (folio) at (-0.4, -0.2) {FOLIO};
\node[process] (optionizer) at (2.35, 0.4) {Optionizer};

% === CENTER-LEFT: SFT ===
\node[model] (sft) at (5.5, -0.4) {SFT};
\node[model] (pi0) at (5.5, 0.8) {$\pi_0$};

% === CENTER: OaK Loop ===
\begin{scope}[shift={(11.5, 0.4)}]
    % Loop background
    \node[loopbox, minimum width=9.5cm, minimum height=3.8cm] (loopbg) at (0, 0) {};
    \node[label, anchor=north] at (0, 2.2) {\textbf{Micro \oak{} Loop} (repeat $N{=}2$ iterations)};
    
    % Loop nodes
    \node[process] (generate) at (-3.1, 0.8) {Generate\\Traces};
    \node[process] (verify) at (-0.5, 0.8) {Solver\\Verify};
    \node[process] (qhat) at (2.4, 0.8) {Update\\$\qhat$};
    \node[process] (prefs) at (-1.6, -1.1) {Build\\Preferences};
    \node[process] (dpo) at (1.6, -1.1) {DPO};
    
    % Loop arrows
    \draw[arrow] (generate) -- (verify);
    \draw[arrow] (verify) -- (qhat);
    \draw[arrow] (qhat.south) -- ++(0, -0.3) -| (prefs.north);
    \draw[arrow] (prefs) -- (dpo);
    \draw[looparrow] (dpo.east) -- ++(0.8, 0) |- (generate.east);
\end{scope}

% === RIGHT: Outputs (aligned on a common x, tied to loop)
\node[right=2.6cm of loopbg.east] (output_anchor) {};
\node[output] (pistar) at (output_anchor |- pi0) {Aligned $\pi^*$};
\node[output] (qhatstar) at (output_anchor |- sft) {Calibrated $\qhat^*$};

% === Solver (external, positioned relative to loop) ===
\node[data, fill=red!8, below=1.0cm of prefs] (solver) {FOL Solver (Z3)};

% === Main Flow Arrows (all relative) ===
\draw[arrow] (prontoqa.east) -- ++(0.3, 0) |- (optionizer.west);
\draw[arrow] (folio.east) -- ++(0.3, 0) |- (optionizer.west);
\draw[arrow] (optionizer.south) |- node[pos=0.7, above, label, yshift=2pt] {traces} node[pos=0.7, below, font=\tiny, text=black!50] {proof $\rightarrow$ Thought/Action} (sft.west);
\draw[arrow] (sft) -- (pi0);
\draw[arrow] (pi0.east) -- ++(1.0, 0) |- (generate.west);

% Arrow from loop to outputs (relative to nodes)
\draw[arrow] (loopbg.east |- pistar) -- (pistar.west);
\draw[arrow] (loopbg.east |- qhatstar) -- (qhatstar.west);

% Solver connection (relative to nodes)
\draw[arrow, black!50] (solver.north) -- (prefs.south);

% === Labels (relative to nodes) ===
\node[label, above=0.8cm of prontoqa] {\textbf{Data}};
\node[label, above=0.5cm of pi0] {\textbf{Init}};
\node[label, above=0.3cm of pistar] {\textbf{Output}};

% Annotations (relative to nodes)
\node[font=\tiny, text=black!50, below=0.5cm of sft] {format learning};

\end{tikzpicture}
}%
\caption{\sokrates{} architecture. \textbf{Left:} PrOntoQA/FOLIO problems are converted to optionized Thought/Action traces. \textbf{Center:} After SFT initialization, the micro \oak{} loop iterates: (1) generate traces from current policy, (2) verify each step with the FOL solver, (3) update the option-success predictor $\qhat$, (4) construct preference pairs, and (5) apply \dpo{} to improve the policy. \textbf{Right:} Final outputs are an aligned policy $\pi^*$ and a calibrated knowledge head $\qhat^*$.}
\label{fig:architecture}
\end{figure*}
